%%%%%%%%%%%%%%%%%%%%%%%%%%%%%%%%%%%%%%%%%%%%%%%
%% Sample Dissertation, Thesis, or Document  %%
%%            for use with the               %%
%%  University of Arizona Thesis Class,      %%
%%               uathesis.cls                %%
%% ----------------------------------------- %%
%%%%%%%%%%%%%%%%%%%%%%%%%%%%%%%%%%%%%%%%%%%%%%%

%% HOW TO USE THIS DOCUMENT:
%% Any line beginning with `%%' is a descriptive comment. DO NOT GET RID OF `%%', 
%% or you will face unresolvable and incomprehensible compiler errors.
%% Any line beginning with `%' is a optional code chunk. Uncomment it by removing
%% `%' and run the code. Please read the associated comments carefully.

%% INTRODUCTION:
%% This is an adapted version of J.J. Charfman's template which can be
%% found here: https://da.overleaf.com/latex/templates/university-of-arizona-astronomy-thesis-template/tsfqcgnfmjcx
%% This template is adapted for linguistics, with support for XeLaTeX, duly tested on . The uncommented line
%% below will produce a Dissertation, the others would produce a Thesis
%% or a Document.  There are other options available to you like turning
%% on the copyright statement and replacing the year on the title page
%% with a "generated on" stamp (handy for early drafts).  To find out
%% what the available options are, take a look into the uathesis.cls
%% file and look for the \DeclareOption commands near the top of that
%% file.
%% There are five copyright options.  Copyright, no copyright, and three
%% different Creative Commons licences.  Use the one you want (If you go
%% Creative Commons, I (DM) think the CC-BY-ND makes the most sense)  See
%% uathesis.cls for the reason why the non-commercial licenses are not
%% included.

\documentclass[dissertation]{uathesis} 
%\documentclass[dissertation,copyright]{uathesis}
%\documentclass[dissertation,CC-BY]{uathesis}
%\documentclass[dissertation,CC-BY-SA]{uathesis}
%\documentclass[dissertation,CC-BY-ND]{uathesis}
%\documentclass[thesis]{uathesis}
%\documentclass[document]{uathesis}

%% ############ Package Usage ############
%% These are the packages that I recommend:

\usepackage{fancyhdr} %% cosmetics for better headings
%% ----This is an attempt to fix the header issue pointed out by Grad College- 
%% comment this section out if its not needed.
    \fancypagestyle{plain}{
        \lhead{}
        \chead{}
        \fancyhead[R]{\thepage}
        \fancyhead[L]{}
        \fancyhead[C]{}
        \renewcommand{\headrulewidth}{0pt}
        \fancyfoot{}
        \setlength{\footskip}{4.35004pt}
    }
    \setlength{\headheight}{20.5pt}
    \pagestyle{plain}
    
    \fancypagestyle{fancy}{
        \fancyhead[R]{\thepage}
        \fancyhead[L]{\leftmark}
        \fancyhead[C]{}
        \renewcommand{\headrulewidth}{0pt}
        \fancyfoot{}
    }

%% This is an attempt at juryrigging an appendix
\usepackage{pdfpages}

\usepackage{xcolor} %% broader color range and color options, 
		    %% allows you to define your own colors as RBG vectors

%% ############ Packages for Tables and Graphics ############
\usepackage{placeins} %% helps with floats and placing elements within sections
\usepackage{float}    %% trying to get my images where I want them

\usepackage{graphicx} %% allows insertion of pictures etc
    \graphicspath{{images/}}

%%  Packages for Tables:
\usepackage{tabularx}
\usepackage{booktabs}
\usepackage{lscape}    %% Used for making fitting large tables in by putting them landscape
\usepackage{colortbl}  %% tables with color class
\usepackage[tableposition=t]{caption} %% (makes table captions normal size when font set smaller)
\usepackage{longtable} %% trying to allow tables to go onto next page
\usepackage{array}     %% trying to control table cell size

%% Package for improving lists:
\usepackage{enumitem}  %% allows advanced controls of tables and example numbering

%% Packages for additional formatting and support:
\usepackage{amsmath}  %% amsmath allows the use of \eqref{}, 
                      %% eqref will automatically put 
                      %% brackets around your reference, 
                      %% which is great for referencing gb4e examples.
\usepackage{siunitx}  %% Required for aligning numbers in a table or equation 
		      %% by decimal and digits, instead of characters.
		      %% Example: 
		      %% 10.591  vs  10.591
		      %%  0.59   vs    0.59
\sisetup{
  round-mode      = places, %% Rounds numbers
  round-precision = 2 	    %% to 2 places
} 

%%############ FONT SUPPORT ############
%% LaTeX offers several compilers, that need their own packages to run. This tempalte provides support for
%% PDFLaTeX and XeLaTeX. If you use TeX or LuaLaTeX, your document will not compile.
%% Choose one of the two following sections, and comment the other: 

%% 1. PDFLaTeX FONT SUPPORT:
%% Uncomment the following if your compiler is PDFLATEX, 
%% and comment the `XELATEX FONT SUPPORT' section:

% \usepackage[utf8]{inputenc} %% inputenc allows you to directly input utf-8 characters. 
			      %% This means you can just type Árüz instead of \'Ar\"uz:
                            
%% 2. XELATEX FONT SUPPORT
%% Uncomment the following if your compiler is XELATEX, and comment the `PDFLATEX FONT SUPPORT' option.
%% Fontspec is a package for loading fonts.
%% To add a new font and call it alongside the main font of the text, 
%% use the format '\newfontfamily\nameofcommand{Font Name}' as shown below.

\usepackage{times}  		   %% enable if you want to use a times font
\usepackage{fontspec}
    % \setmainfont{Times} 	   %% Uncomment if you want to use Times
    \newfontfamily\ipa{Doulos SIL} %% ALT FONT for IPA symbols. Ensure system has font.

%% ############ Packages for Citations: ############                          
%% LaTeX has two bibliography compiling systems, natbib and biber. Choose the one that you intend to use
%% and comment the other:

%% 1. NATBIB: Comment (2) and run the following:
%% natbib is available on most systems, and is terribly handy.
%% We are using biber for this template, but leaving natbib in as an option
%% if you want to use natbib instead of biber/biblatex, you need to 
%% comment out the next package and the \addbibresource params:

% \usepackage{natbib}
% \setcitestyle{semicolon,aysep={},yysep={,},notesep={:}}

%% 2. BIBER: Comment (1) and run the following:
%% Biber is the second option. 
\usepackage[backend=biber,
                style=unified, %% we are using the LSA unified style sheet here
                maxcitenames=2,
                maxbibnames=99,
                natbib]{bib latex}
\addbibresource{bibliography.bib}
\renewcommand{\bibname}{References}

%% ###################### TODONOTES #######################
%% Use one of the following two package invocations. 
%% The first is for the final version, the second for adding todonotes and comments:

%% 1. DISABLE TODONOTES, comment code in (2)
%\usepackage[disable]{todonotes}

%% 2. ENABLE TODONOTES, comment code in (1)
\usepackage[colorinlistoftodos,backgroundcolor= yellow, bordercolor= black]{todonotes}
    \setuptodonotes{inline, prepend, caption={ToDo}}

%% ################### FORMATTING HYPERLINKS ###############
%% If you are using hyper-ref (recommended), this command must go after all 
%% other package inclusions (from the hyperref package documentation).
%% The purpose of hyperref is to make the PDF created extensively cross-referenced.
%% It allows us to use references as links [hidelinks], 
%% makes sure there are no borders around the link, and gives us \autoref:

\usepackage[hidelinks]{hyperref}

%% ########## Linguistics Packages ############
\usepackage{tikz}	%% graph solution that should help me get a nice vowel chart
\usepackage{tipa}	%% enables an IPA environment
\usepackage{qtree}	%% allows simple trees
\usepackage[linguistics]{forest} %% allows more complex trees
\usepackage{gb4e} %% manages example numbering, has to be last package in pre-amble, will break document otherwise

%% ########## Set up values for Title Page ############
 
\completetitle{Title}		   %% Title of your dissertation
\fullname{}			   %% Grad college wants your full name here.
\degreename{Doctor of Philosophy}  %% Title of your degree.
\degreemajor{Linguistics}          %% Degree major
\counterwithout{footnote}{chapter} %% disable this if you want your footnotes to reset their counter for each chapter

%% the inlinetitle below will be used on the graduate college signature page if you uncomment it. 
%% This is useful if your completetitle uses a linebreak or other command that would break an inline use of that title. 
%% If you leave this commented it will do nothing. Uncomment and give it a value to activate. The change will be visible on page 2.

% \inlinetitle{}

%% ########## BEGIN DOCUMENT ############
\begin{document}

% this file can host your custom commands so they do not clutter your mainfile

% for documentation on LaTeX commands see: https://www.overleaf.com/learn/latex/Commands#Simple_commands


%%% The following allows dagger footnote on chapter titles
%%% with no number. Useful for designating previously 
%%% published work.
\newcounter{daggerfootnote}
\newcommand*{\daggerfootnote}[1]{%
    \setcounter{daggerfootnote}{\value{footnote}}%
    \renewcommand*{\thefootnote}{\fnsymbol{footnote}}%
    \footnote[2]{#1}%
    \setcounter{footnote}{\value{daggerfootnote}}%
    \renewcommand*{\thefootnote}{\arabic{footnote}}%
    }


%% Custom ToDo notes
%%  Useage: \done{completed todo text, prints 'done'}, \nw{todos from nw}, \todocite{prints the work cite}
%% todo notes for completed todos:
    \newcommand{\done}[1]{\todo[color=green!40, prepend, caption={Done}]{#1}}
%% todo notes by self:
    \newcounter{todocounter}
    \newcommand{\todonum}[2][]
    {\stepcounter{todocounter}\todo[#1]
    {\thetodocounter: #2}}
%% todo notes by NW
    \newcommand{\todoNW}[2][]
    {\stepcounter{todocounter}\todo[color=orange!40, author=NW, noprepend]{\thetodocounter: #2}}   

%% todo notes for adding citation
    \newcommand{\todocite}[2][]
    {\stepcounter{todocounter}\todo[inlinewidth=2.2cm, noinlinepar, color=blue!40, prepend, caption={CITE}]
    {\thetodocounter: #2}}  

%% todo notes for reverse outline
     \newcommand{\reverseOutline}[2][]
     {\stepcounter{todocounter}\todo[color=grey!40, noprepend]{\thetodocounter: #2}}

 %% we are hosting all the custom commands we are writing in a seperate file

%% ########## Set up depth values to number sections ############
%% tocdepth defines how many layers of sections/subsections are displayed in your table of content. The hierarchy works like this:
%% Chapters (1)> Sections (2) > Subsections (3) > SubSubsections (4) > Paragraphs (5)
%% Default = 4
%% secnumdepth does the same work, except it numbers your sections. 
%% Chapters are automatically numbered, so setting this to 4 will give you numbered paragraphs. 
%% Default = 3

% \setcounter{tocdepth}{5} 
% \setcounter{secnumdepth}{4}



%% Set up the title page
\maketitlepage
{DEPARTMENT OF LINGUISTICS}	%% Title of your department.
{2019}							

%% Insert the approval form.  Note that for electronic submission
%% of your Ph. D. dissertation, you must bring *two* copies of the
%% approval page to your final defense.  These must be signed by
%% the committee.  Make two photocopies: one for the department
%% and the other for your records.  Then, bring the two signed 
%% originals to the graduate college when you submit the 
%% final version of the dissertation to the University of Arizona.
%% if you want to change the number of lines displayed, 
%% you need to do this in the uathesis.cls class, check the \approval command

\approval
{}	%% Defense Date	
{}	%% Dissertation Director
{}	%% 1st committee member
{}	%% 2nd committee member
{}	%% 3rd committee member
{}	%% 4th committee member
{}	%% 5th committee member
{}	%% 6th committee member

%% if you want to load your approval with a UA watermark, use this command, which uses the same syntax as the one above:
%% \approvalWithWM{}{}{}{}{}{}{}{}

%% Include the ``Statement by Author'' for Dissertations
\statementbyauthor

%% If this is a Thesis, use the following form, with your thesis director's
%% name and title in the square brackets like so (you should also omit the 
%% approval form insertion above):

%\statementbyauthor[Jane M. Doe\\Professor of Chemistry]

%% Include the ``Acknowledgements''
\incacknowledgements{sections/acknowledgements}

%% Include the ``Dedication''
\incdedication{sections/dedication}

%% Create a ``Table of Contents''
\tableofcontents

%% Create a ``List of Figures''
\listoffigures

%% Create a ``List of Tables''
\listoftables

%% Include the ``Abstract''
\incabstract{sections/abstract}

%% Include the various chapters
%!TEX root = ../dissertation.tex

\chapter[Introduction]% the [square brackets] define the name of this in the TOC
{Introduction\daggerfootnote{This chapter has been published previously as \citet{Chimsky2022}.}}\label{chap:Intro} %we define a label for this chapter

Here are different cite commands as demonstrated in a \emph{gb4e} list:

\begin{exe}
    \ex \textbackslash cite : \cite{chomsky-why-2016}
    \ex now here is a sublist \begin{xlista}
    \ex \textbackslash citep : \citep{chomsky-why-2016}
    \ex \textbackslash citet{} : \citet{chomsky-why-2016}
    \end{xlista}
    \ex\label{item} now there is another numbered item
\end{exe}
And now I reference that item with \textbackslash eqref: \eqref{item}

This begins a discussion of Chimsky \todocite{}
%\begin{figure}
%\centering
%\includegraphics[angle=0,width=\columnwidth]{fig1.pdf}
%\caption[]{}
%\label{fig1}
%\end{figure}

%!TEX root = ../dissertation.tex

\chapter[Chapter 2]{TitleofChapter}

Let's make a figure out of a tree and position it right under our text:

\begin{figure}[h] %to understand figure positioning, please refer to: 
% https://www.overleaf.com/learn/latex/Inserting_Images
    \centering
    \begin{forest}
    [NP [DP] [N] ]
    \end{forest}
    \caption[A very special figure]{Caption}
    \label{fig:mylabel}
\end{figure}
\todoNW{Example Todo Note from NW}

Here is a table positioned at the top of the page, with the `S' column used to align numerical data by decimal points. Compare the second column to the third and fourth. Notice how text is enclosed in \\text\{\}, as `s' type columns read all characters as numbers, and will error out if other characters are not properly declared:

\begin{table}[t]
\centering
\footnotesize
\begin{tabularx}{0.85\textwidth}{llSS}
\toprule
Col 1 & Col 2 & \text{Numerical column 3} &  \text{Numerical column 3} \\
\midrule
1 & 124   & 3              & 24.82   \text{\$} \\ 
2 & 463   & 1000           & 0       \text{\$} \\ 
3 & 675   & 564            & 0       \text{\$} \\ 
4 & 175   & 33             & 241.139 \text{\$} \\ 
5 & 357   & 56             & 13.5    \text{\$} \\ 
6 & 15685 & 253            & 2.67    \text{\$} \\ 
7 & 23    & \text{unknown} & 27.27   \text{\$} \\ 
\bottomrule
\end{tabularx}
\caption{Example of numerical table}
\label{tab:numerical-table}
\end{table}

And here is a table which we position at the bottom:
%% some custom definitions first
\newcolumntype{s}{>{\columncolor[HTML]{AAACED}} p{3cm}} 
%% here we define a custom column
%% type, called `s', this will have the color as defined via HTML and a width of 3cm

\arrayrulecolor[HTML]{DB5800}
\colorlet{mycolor}{red!30!blue!50} %% here we define our own color, using red  
                                   %% at 30% opacity and blue at 50%
%%%%%%%%% https://www.overleaf.com/learn/latex/Tables
\begin{table}[b!]
\begin{tabular}{ |s|p{3cm}|p{3cm}| }
\hline
\rowcolor{lightgray} \multicolumn{3}{|c|}{Country List} \\
\hline
Country Name or Area Name \cellcolor{mycolor}
& ISO ALPHA 2 Code &ISO ALPHA 3 \\
\hline
Afghanistan & AF &AFG \\
\rowcolor{gray}
Aland Islands & AX & ALA \\
Albania   &AL & ALB \\
Algeria  &DZ & DZA \\
American Samoa & AS & ASM \\
Andorra & AD & \cellcolor[HTML]{AA0044} AND    \\
Angola & AO & AGO \\
\hline
\end{tabular}
\caption[the table's list reference in the list of tables]{A table with some color, positioned at the bottom of a page}
\end{table}

%!TEX root = ../dissertation.tex

\chapter[ChapterShortTitle]{ChapterTitle}

Now let's walk through some \textbackslash ref commands:

\begin{exe}
    \ex \textbackslash ref: \ref{chap:Intro}
    \ex \textbackslash autoref: \autoref{chap:Intro}
    \ex \textbackslash eqref: \eqref{chap:Intro}
    \ex \textbackslash nameref: \nameref{chap:Intro}
    
\end{exe}

\todo{an example of the default todo note}



%% Include the various appendices
\appendix
%!TEX root = ../dissertation.tex

\noindent Add acknowledgements here. 
\todonum{Complete this text.}
\chapter{Extra Stuff}

%\begin{figure}
%\centering
%\includegraphics[angle=0,width=\columnwidth]{fig1.pdf}
%\caption[]{}
%\label{fig1}
%\end{figure}


%% ########## Set up BIBLIOGRAPHY ############
%% Switch the spacing to single-spaced for the references
\renewcommand{\baselinestretch}{1}	%% changing the value
\small\normalsize			%% switch size to make the value take

%% 1.  Use the below if using natbib, comment code in (2):
%% this is where your natbib stylesheet is referenced:
% \bibliographystyle{sp-lsa.bst}	%% S\&P bibliography style, an LSA publication,
					%% meets the LSA guidelines.  
					%% if you wish to use the UA citation style instead, 
     					%% call "uabibnat.bst" in the \bibliographystyle
% \bibliography{bibliography}		%% Call bibliography file, with the .bib extension

%% 2. Use this for biblatex/biber, comment code in (1):
%% 'title = ' changes the heading of the bibliography section,
%% and 'heading = ' ensures it is listed in the ToC 

\printbibliography[title = {REFERENCES}, heading=bibintoc]

\end{document}

%!TEX root = ../dissertation.tex

\chapter[Chapter 2]{TitleofChapter}

Let's make a figure out of a tree and position it right under our text:

\begin{figure}[h] %to understand figure positioning, please refer to: 
% https://www.overleaf.com/learn/latex/Inserting_Images
    \centering
    \begin{forest}
    [NP [DP] [N] ]
    \end{forest}
    \caption[A very special figure]{Caption}
    \label{fig:mylabel}
\end{figure}
\todoNW{Example Todo Note from NW}

Here is a table positioned at the top of the page, with the `S' column used to align numerical data by decimal points. Compare the second column to the third and fourth. Notice how text is enclosed in \\text\{\}, as `s' type columns read all characters as numbers, and will error out if other characters are not properly declared:

\begin{table}[t]
\centering
\footnotesize
\begin{tabularx}{0.85\textwidth}{llSS}
\toprule
Col 1 & Col 2 & \text{Numerical column 3} &  \text{Numerical column 3} \\
\midrule
1 & 124   & 3              & 24.82   \text{\$} \\ 
2 & 463   & 1000           & 0       \text{\$} \\ 
3 & 675   & 564            & 0       \text{\$} \\ 
4 & 175   & 33             & 241.139 \text{\$} \\ 
5 & 357   & 56             & 13.5    \text{\$} \\ 
6 & 15685 & 253            & 2.67    \text{\$} \\ 
7 & 23    & \text{unknown} & 27.27   \text{\$} \\ 
\bottomrule
\end{tabularx}
\caption{Example of numerical table}
\label{tab:numerical-table}
\end{table}

And here is a table which we position at the bottom:
%% some custom definitions first
\newcolumntype{s}{>{\columncolor[HTML]{AAACED}} p{3cm}} 
%% here we define a custom column
%% type, called `s', this will have the color as defined via HTML and a width of 3cm

\arrayrulecolor[HTML]{DB5800}
\colorlet{mycolor}{red!30!blue!50} %% here we define our own color, using red  
                                   %% at 30% opacity and blue at 50%
%%%%%%%%% https://www.overleaf.com/learn/latex/Tables
\begin{table}[b!]
\begin{tabular}{ |s|p{3cm}|p{3cm}| }
\hline
\rowcolor{lightgray} \multicolumn{3}{|c|}{Country List} \\
\hline
Country Name or Area Name \cellcolor{mycolor}
& ISO ALPHA 2 Code &ISO ALPHA 3 \\
\hline
Afghanistan & AF &AFG \\
\rowcolor{gray}
Aland Islands & AX & ALA \\
Albania   &AL & ALB \\
Algeria  &DZ & DZA \\
American Samoa & AS & ASM \\
Andorra & AD & \cellcolor[HTML]{AA0044} AND    \\
Angola & AO & AGO \\
\hline
\end{tabular}
\caption[the table's list reference in the list of tables]{A table with some color, positioned at the bottom of a page}
\end{table}
